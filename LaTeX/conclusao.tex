\documentclass[12pt]{article}
%encoding
%--------------------------------------
\usepackage[T1]{fontenc}
\usepackage[utf8]{inputenc}
%--------------------------------------

%Portuguese-specific commands
%--------------------------------------
\usepackage[portuguese]{babel}
%--------------------------------------

%Hyphenation rules
%--------------------------------------
\usepackage{hyphenat}
\hyphenation{mate-mática recu-perar}
%--------------------------------------
\usepackage{enumerate}

\usepackage{hyperref}
\hypersetup{
    colorlinks=true,
    linkcolor=blue,
    urlcolor=blue,
}

\urlstyle{same}

\begin{document}

\section{Conclusão}

O desenvolvimento desse trabalho evidenciou alguns dos principais aspectos positivos e negativos identificados no processo seletivo da \emph{Conduent}, e, de maneira geral, pode-se constatar que há nele sinais da crescente centralidade no indivíduo adotada por empresas no mundo contemporâneo, porém, ainda existem algumas fragilidades nas etapas que não proporcionam total valorização do indivíduo.

Observa-se que, ao disponibilizar vagas online, a empresa conseguiu atrair um público diverso, mas notou-se a falta de estrutura para um grupo que ainda enfrenta diversas dificuldades de inserção no mercado de trabalho: as Pessoas Com Deficiência (PCD). Dado o ideal de diversidade preconizado pela organização, essa inclusão mostra-se benéfica por contribuir para que mais indivíduos tenham oportunidades no mercado de trabalho. 

Já em relação aos testes realizados do início ao fim da entrevista, tem-se que havia neles algumas disparidades em relação ao real nível de conhecimento demandado pela função de Analista de Suporte Bilíngue Júnior - ora o conhecimento era menor, ora maior que o necessário - e a falta de testes e dinâmicas indivíduais e/ou em grupo que pudessem demonstrar melhor o perfil psicológico do candidato. Essa análise acabou por ficar restrita a uma única entrevista presencial e, dado que a função demandava um bom nível de manipulação das \emph{Soft Skills} também abordadas nesse trabalho, existe o risco de que as expectativas e potencial do candadito poderiam encontrar-se desniveladas, causando, assim, frustração ou insatisfação nesse.

No que tange o treinamento após aprovação nas etapas anteriores, focava-se no autodesenvolvimento das habilidades técnicas e emocionais de cada indivíduo e buscava mostrar a esses as oportunidades de crescimento que a empresa fornecia. Nesse ponto, o conforto gerado pelo ambiente criado caracteriza-se como um dos principais pontos positivos do processo, pois durante ele, a centralidade no indivíduo se manisfestou da maneira visível. Contudo, a prova final, ainda seguia um modelo pré-pronto e, isolada, desconsiderava aspectos psicológicos do indivíduo, demonstrando que, apesar de grande valorização do humano durante o perído de treinamento, a habilidade técnica por vezes teria maior peso. 

Diante de tais pontos-chave e dos demais abordados no trabalho, conclui-se que o processo de seleção da  \emph{Conduent} do Brasil ainda não está integralmente voltado ao indivíduo, e tal como muitas empresas, necessita de readaptações e reformulações de etapas e testes para que isso ocorra. 

As maneiras pelas quais essas reestruturações podem ocorrer são diversas, mas observa-se um ponto em comum que poderia ser de grande auxilio para isso: uma maior diversificação na análise das \emph{Hard Skills} e personalização na análise do perfil psicológico dos candidatos, sempre objetivando entender se a função contribuirá para que esse alcance autorrealização.

Assim, ao ter-se maior enfoque na compreensão de se a empresa encaixa-se no perfil e objetivos do candidato, temos que frustrações são evitadas por ambas as partes e a nitidez na relação entre empregador e emprego é estabelecida desde seu início. 

Um exemplo de personalização e diversificação adotado por algumas empresas centradas no indíviduo é a aplicação de um treinamento antes mesmo da análise elimitória de currículos. 

Dar a cada indivíduo interessado a oportunidade de demonstrar suas capacidades em maior tempo tanto de habilidades técnicas quanto interpessoais mostra-se como uma alternativa justa ao modelo de entrevista ainda predominante que valoriza as individualidades de cada candidato e facilita o seu autorrealizar individual. 

Isto posto, conclui-se que a área de recrutamento é dinâmica e tende criar mais ferramentas e alternativas para que o ser humano seja cada vez mais valorizado. O que torna necessário e benéfico o consequente enfoque das organizações em cada um dos indivíduos que integram seu time. 


\end{document}